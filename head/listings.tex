\usepackage{calc}
\usepackage{listings}
\usepackage{xcolor}
\usepackage{scalefnt}

\makeatletter

\lstset{%
  captionpos=b,
  tabsize=2,
  numbers=left,
  numbersep=5pt,
  basicstyle=\ttfamily\lst@ifdisplaystyle\scalefont{.9}\fi,
  numberstyle=\ttfamily\scalefont{.9},
  keywordstyle=\color{RoyalBlue},
  commentstyle=\color{green!50!black},
  stringstyle=\color{red},
  keepspaces=true,
  showstringspaces=false,
  breaklines=true,
}

% ----- BOOK-STYLE LISTINGS - COMMENT FOR NORMAL LISTINGS -----
% For fancy listing captions, we need to load the caption package, which will
% override the template style for tables and figures. The following four steps
% will circumvent this issue.

% 1. Store original caption command before it gets overridden by caption package
\let\originalmakecaption\@makecaption

% 2. Load caption package
\usepackage{caption}

% 3. Store caption packages caption command before defining our own
\let\captionsmakecaption\@makecaption

% 4. Define our own caption command
\long\def\@makecaption#1#2{%
  \ifthenelse{\equal{\@captype}{lstlisting}}{%
    % This is a listing, use caption packages caption command
    \captionsmakecaption{#1}{#2}%
  }{%
    % This is not a listing, use the templates original caption command
    \originalmakecaption{#1}{#2}%
  }
}

\definecolor{captiongray}{rgb}{0.56470, 0.56470, 0.56470}

\newlength{\captionskip}
\newlength{\captionlength}
\DeclareCaptionFont{white}{\color{white}}
\DeclareCaptionFormat{listing}{%
  \settodepth{\captionskip}{g}%
  \settowidth{\captionlength}{#1#2}%
  \colorbox{captiongray}{%
    \hspace{.5\captionskip}%
    \parbox{\linewidth-.85em-\captionskip}{%
      \parbox[t]{\captionlength}{\strut#1#2}%
      \parbox[t]{\linewidth-.85em-\captionskip-\captionlength}{#3}%
      \vspace{-\captionskip}} % vertically align
    \hspace{.5\captionskip}}}

\captionsetup[lstlisting]{%
  format=listing,%
  singlelinecheck=false,%
  margin=0pt,%
  font={white,sf,bf,small},%
  labelsep=colon}

\lstset{%
  frame=b,
  captionpos=t,
  basicstyle=\ttfamily\lst@ifdisplaystyle\scalefont{.9}\fi,
  numberstyle=\ttfamily\scalefont{.9},
  keywordstyle=\color{RoyalBlue},
  commentstyle=\color{green!50!black},
  stringstyle=\color{red},
  rulecolor=\color{captiongray},
  aboveskip=0pt,
  belowskip=0pt,
  backgroundcolor=\color{white},
  belowcaptionskip=3pt,
  xleftmargin=17pt,
  framexleftmargin=17pt,
  framexrightmargin=0pt,
  framexbottommargin=0pt,
}
% ----- END OF BOOK-STYLE LISTINGS - COMMENT FOR NORMAL LISTINGS -----

\makeatother

\lstdefinelanguage{HIPAcc}{%
  language=C++,
  morekeywords={size_t, uchar, ushort, uint,
    uchar2, char2, ushort2, short2, uint2, int2, float2, double2,
    uchar4, char4, ushort4, short4, uint4, int4, float4, double4,
    uchar8, char8, ushort8, short8, uint8, int8, float8, double8,
    uchar16, char16, ushort16, short16, uint16, int16, float16, double16},
  keywords=[2]{Image, Accessor, Kernel, IterationSpace, Mask, BoundaryCondition,
    Boundary, CLAMP, MIRROR, UNDEFINED, REPEAT, CONSTANT, output, convolve,
    iterate, reduce, Reduce, SUM, MIN, MAX, PROD, MEDIAN Domain},
  keywordstyle=[2]{\color{red!50!black}\bfseries}
}

\lstdefinestyle{CUDA}{%
  language=C++,
  morekeywords={size_t, uchar, ushort, uint,
    uchar2, char2, ushort2, short2, uint2, int2, float2, double2,
    uchar4, char4, ushort4, short4, uint4, int4, float4, double4,
    uchar8, char8, ushort8, short8, uint8, int8, float8, double8,
    uchar16, char16, ushort16, short16, uint16, int16, float16, double16},
  otherkeywords=[2]{>>>,<<<,[,]},
  alsoletter=[2]{.},
  keywords=[2]{__device__, __host__, __global__, __constant__, __shared__, dim3,
    blockIdx.x, blockIdx.y, blockDim.x, blockDim.y, blockDim.z, threadIdx.x,
    threadIdx.y, threadIdx.z, gridDim.x, gridDim.y, warpSize, __syncthreads,
    __sinf, __cosf, __expf, cudaMalloc, cudaMemset, cudaFree, cudaMemcpy,
    cudaMemcpyHostToDevice, cudaMemcpyDeviceToHost, cudaMemcpyDeviceToDevice,
    tex1Dfetch, tex2D, surf2Dwrite},
  keywordstyle=[2]{\color{blue}\bfseries}
}

\lstdefinestyle{OpenCL}{%
  language=C,
  morekeywords={size_t, uchar, ushort, uint,
    uchar2, char2, ushort2, short2, uint2, int2, float2, double2,
    uchar4, char4, ushort4, short4, uint4, int4, float4, double4,
    uchar8, char8, ushort8, short8, uint8, int8, float8, double8,
    uchar16, char16, ushort16, short16, uint16, int16, float16, double16},
  keywords=[2]{__kernel, __global, __constant, __local, get_num_groups,
    get_local_size, get_group_id, get_local_id, get_global_id, get_global_size,
    barrier, mem_fence, CLK_GLOBAL_MEM_FENCE, CLK_LOCAL_MEM_FENCE,
    read_mem_fence, write_mem_fence, read_imagef, write_imagef},
  keywordstyle=[2]{\color{red!50!black}\bfseries}
}

