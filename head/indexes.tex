% Match label commands from acronyms package to glossaries package.
% (Commands mark acronym used, first use produces page number in acronyms list)
\newcommand{\acresetall}{\glsresetall}
\newcommand{\ac}[1]{\gls{#1}}
\newcommand{\acf}[1]{\acrfull{#1}\glsunset{#1}}
\newcommand{\acs}[1]{\acrshort{#1}\glsunset{#1}}
\newcommand{\acl}[1]{\acrlong{#1}\glsunset{#1}}
\newcommand{\acp}[1]{\glspl{#1}}
\newcommand{\acfp}[1]{\acrfullpl{#1}\glsunset{#1}}
\newcommand{\acsp}[1]{\acrshortpl{#1}\glsunset{#1}}
\newcommand{\aclp}[1]{\acrlongpl{#1}\glsunset{#1}}
\newcommand{\acfi}[1]{\emph{\acrlong{#1}} (\textsc{\acrshort{#1}})\glsunset{#1}}
\newcommand{\iac}[1]{a \ac{#1}}
\newcommand{\Iac}[1]{A \ac{#1}}
\newcommand{\Ac}[1]{\Gls{#1}}
\newcommand{\Acf}[1]{\Acrfull{#1}\glsunset{#1}}
\newcommand{\Acs}[1]{\Acrshort{#1}\glsunset{#1}}
\newcommand{\Acl}[1]{\Acrlong{#1}\glsunset{#1}}
\newcommand{\Acp}[1]{\Glspl{#1}}
\newcommand{\Acfp}[1]{\Acrfullpl{#1}\glsunset{#1}}
\newcommand{\Acsp}[1]{\Acrshortpl{#1}\glsunset{#1}}
\newcommand{\Aclp}[1]{\Acrlongpl{#1}\glsunset{#1}}
\newcommand{\Acfi}[1]{\emph{\Acrlong{#1}} (\textsc{\Acrshort{#1}})\glsunset{#1}}
\newcommand{\iAc}[1]{a \Ac{#1}}
\newcommand{\IAc}[1]{A \Ac{#1}}

% Don't mark acronym (#1) used when calling command (#2)
\newcommand{\dontmarkused}[2]{%
  \ifglsused{#1}{%
    % Acronym was used before, just call the command
    #2%
  }{%
    % Set used (avoids entry in acronyms list) and reset afterwards
    \glsunset{#1}%
    #2%
    \glsreset{#1}%
  }}

% Match label commands with star from acronyms package to glossaries package
% (Commands don't mark acronym used, won't produce page number in acronyms list)
\usepackage{suffix}
\WithSuffix\newcommand\acf*[1]{\dontmarkused{#1}{\acf{#1}}}
\WithSuffix\newcommand\acs*[1]{\dontmarkused{#1}{\acs{#1}}}
\WithSuffix\newcommand\acl*[1]{\dontmarkused{#1}{\acl{#1}}}
\WithSuffix\newcommand\acfp*[1]{\dontmarkused{#1}{\acfp{#1}}}
\WithSuffix\newcommand\acsp*[1]{\dontmarkused{#1}{\acsp{#1}}}
\WithSuffix\newcommand\aclp*[1]{\dontmarkused{#1}{\aclp{#1}}}
\WithSuffix\newcommand\acfi*[1]{\dontmarkused{#1}{\acfi{#1}}}
\WithSuffix\newcommand\ac*[1]{\ifglsused{#1}{\ac{#1}}{\acf*{#1}}}
\WithSuffix\newcommand\acp*[1]{\ifglsused{#1}{\acp{#1}}{\acfp*{#1}}}
\WithSuffix\newcommand\iac*[1]{a \ac*{#1}}
\WithSuffix\newcommand\Iac*[1]{A \ac*{#1}}
\WithSuffix\newcommand\Acf*[1]{\dontmarkused{#1}{\Acf{#1}}}
\WithSuffix\newcommand\Acs*[1]{\dontmarkused{#1}{\Acs{#1}}}
\WithSuffix\newcommand\Acl*[1]{\dontmarkused{#1}{\Acl{#1}}}
\WithSuffix\newcommand\Acfp*[1]{\dontmarkused{#1}{\Acfp{#1}}}
\WithSuffix\newcommand\Acsp*[1]{\dontmarkused{#1}{\Acsp{#1}}}
\WithSuffix\newcommand\Aclp*[1]{\dontmarkused{#1}{\Aclp{#1}}}
\WithSuffix\newcommand\Acfi*[1]{\dontmarkused{#1}{\Acfi{#1}}}
\WithSuffix\newcommand\Ac*[1]{\ifglsused{#1}{\Ac{#1}}{\Acf*{#1}}}
\WithSuffix\newcommand\Acp*[1]{\ifglsused{#1}{\Acp{#1}}{\Acfp*{#1}}}
\WithSuffix\newcommand\iAc*[1]{a \Ac*{#1}}
\WithSuffix\newcommand\IAc*[1]{A \Ac*{#1}}

\makeatletter
\@ifpackageloaded{biblatex}{%
% Use internals of \MakeSentenceCase*{} command from BibLaTeX:
\@ifpackagewith{biblatex}{casechanger=latex2e}{}{%
  % IMPORTANT: BibLaTeX must be loaded with 'casechanger=latex2e' for
  %            \MakeSentenceCase*{}'s internals to be available.
  \@latex@error{load biblatex package with option 'casechanger=latex2e'}}
% Transforms all characters to lower case except those that were escaped using
% curly braces {}, similar to titles in BibLaTeX entries.
\newrobustcmd{\MakeLowercaseEscaped}{%
  % Similar to \MakeSentenceCase*{}, just without capitalizing first character
  \let\blx@mksc@endhead\relax%
  \blx@mksc@ii}
}{%
% Fallback: Do not transform to lower case!
\newcommand\MakeLowercaseEscaped[1]{#1}
}
\makeatother

% Match newacro command from acronyms package to glossaries package.
% If BibLaTeX package is loaded, the label expands to lower case in document
% body but remains untouched in acronyms list. Proper nouns can be escaped using
% curly braces {} similar to titles in BibLaTeX entries.
\usepackage{xargs}
\newcommandx{\newacro}[4][2=\empty, 4=\empty]{%
  \ifx#4\empty% no plural description
    \ifx#2\empty% no formated acronym
      \newacronym[sort={#1},description={#3}]{#1}{#1}{\MakeLowercaseEscaped{#3}}%
    \else% formated acronym
      \newacronym[sort={#1},description={#3}]{#1}{#2}{\MakeLowercaseEscaped{#3}}%
    \fi%
  \else% plural description
    \ifx#2\empty% no formated acronym
      \newacronym[sort={#1},description={#3},longplural={\MakeLowercaseEscaped{#4}}]{#1}{#1}{\MakeLowercaseEscaped{#3}}%
    \else% formated acronym
      \newacronym[sort={#1},description={#3},longplural={\MakeLowercaseEscaped{#4}}]{#1}{#2}{\MakeLowercaseEscaped{#3}}%
    \fi%
  \fi}

% Match commands from listofsymbols package to glossaries package
\newglossary[slg]{symbols}{sym}{sbl}{List of Symbols\ifx\final\undefined\ (draft)\fi}
\newcommand{\newsym}[3][\empty]{%
  \newglossaryentry{sym:#2}{%
    type=symbols,%
    name=\ensuremath{#3},%
    description={\ifx\final\undefined\texttt{\textbackslash #2} -- \fi#1}%
  }
  \expandafter\newcommand\expandafter{\csname#2\endcsname}%
    {{\gls{sym:#2}}\xspace}
}

% New style for acronyms and symbols
\newglossarystyle{dottedlist}{%
  \renewenvironment{theglossary}%
    {\begin{description}[labelwidth=2.5cm, itemsep=0pt]}{\end{description}}%
  \renewcommand*{\glossaryheader}{}%
  \renewcommand*{\glsgroupheading}[1]{}%
  \renewcommand*{\glossentry}[2]{%
    \item[\glsentryitem{##1}%
          \glstarget{##1}{\glossentryname{##1}}]%
       \glossentrydesc{##1}\dotfill ##2}%
  \renewcommand*{\subglossentry}[3]{%
    \glssubentryitem{##2}%
    \glstarget{##2}{\strut}\space
    \glossentrydesc{##2}\dotfill ##3.}%
  \renewcommand*{\glsgroupskip}{\ifglsnogroupskip\else\indexspace\fi}%
}

% Enable acronyms and glossary without makeindex
\makenoidxglossaries

% Different headings for first index page
\renewcommand*{\indexpagestyle}{scrplain}
\makeindex

% Do not use colors for any glossary link
\makeatletter
\newcommand*{\glsplainhyperlink}[2]{%
  \colorlet{currenttext}{.}% store current text color
  \colorlet{currentlink}{\@linkcolor}% store current link color
  \hypersetup{linkcolor=currenttext}% set link color
  \hyperlink{#1}{#2}%
  \hypersetup{linkcolor=currentlink}% reset to default
}
\let\@glslink\glsplainhyperlink
\makeatother

% Write index style file (taken from KOMA-Script-Anleitung)
\usepackage{filecontents}
\begin{filecontents}{\jobname .mst}
%level  '>'
%actual '='
%encap  '!'
%quote  '~'
%
preamble  "\\clearpage\\begin{theindex}\\begin{small}\n"
postamble "\n\n\\end{small}\\end{theindex}\n"
%
delim_0 "~\\dotfill~"
delim_1 "~\\dotfill~"
delim_2 "~\\dotfill~"
%
heading_prefix "{\\hspace{0pt plus 2fil}\\textbf{"
heading_suffix "}\\hspace{0pt plus 1fil}}\\nopagebreak\n"
headings_flag  1
\end{filecontents}

