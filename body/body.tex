\chapter{Methods}\label{sec:methods}
This Section proposes useful guidelines for typesetting and using this template.

\section{Citations}
Either cite the reference id~\parencite{RKHT17} only, the authors
\citeauthor{RKHT17} only, or both at once \eg \textcite[see][51\psq]{MGBCR04}.
In addition, citations can be placed in
footnotes\footcite[see][50\psqq]{MGBCR04} as well.

\section{References}
The type of reference is automatically resolved, such as \Cref{sec:methods},
\Cref{tab:example}, \Cref{fig:example}, \Cref{fig:subfig1,fig:subfig2},
\Cref{lst:hello}, and \Cref{alg:euclid}.

\section{Symbols, Acronyms, Glossary, Index}
A list of symbols can be defined globally. Afterwards, specifying the user
defined command results in printing the symbol, such as \stddev. You can access
another predefined symbol in a symbol definition \eg \variance (squared symbol
for standard deviation).

The first use of \ac{ALU} is written out followed by its abbreviation for later
use. Therefore, the second use of \ac{ALU} will only put the abbreviation.
Moreover, it is also possible to specify the use of plural if multiple
\acp{ALU} are mentioned in text.

Entries to the glossary can be created by using one of the defined entries \eg
\gls{computer}.

For indexing\index{Indexing} specific keywords\index{Keywords} and
corresponding subkeywords\index{Keywords!Sub}, just tag their use.

\section{Lists}
Besides using \code{itemize} and \code{enumerate}, also \emph{inline} lists can
be defined. Advantages are:
\begin{enumerate*}[label=\upshape(\alph*),itemjoin={{; }},itemjoin*={{; and }}]
  \item Automatic generation of \emph{inline} numbering for each item
  \item insertion of separators between items
  \item placement of a different separator for the last item.
\end{enumerate*}
However, do not forget to put a full stop at the last item.

\section{Tables}
\begin{table}[t]
  \caption{A very simple example table.}
  \label{tab:example}
  \centering
  \scalebox{0.7}{
    \begin{tabular}{llrrrrrrr}
      \toprule
      Filter                    & Type   & II & ALUTs & Registers & LU (\%) & BRAM & DSP & Freq [MHz] \\
      \midrule
      \multirow{2}{*}{Gaussian} & normal &  2 &  8593 &     12423 &   19.86 &   84 &   4 &     132.50 \\
                                & fused  &  2 &  8439 &     11843 &   19.26 &   83 &   3 &     153.11 \\
      \cmidrule{2-9}
      \multirow{2}{*}{Sobel}    & normal &  2 &  8552 &     12433 &   19.76 &   84 &   4 &     131.25 \\
                                & fused  &  2 &  8230 &     12098 &   19.11 &   84 &   3 &     152.09 \\
      \bottomrule
    \end{tabular}
  }
\end{table}

Common rules for tables are to put captions above the table, to entirely omit
vertical lines, to put units in the table's header, to align numbers to the
right, and to always use the same amount of decimal places within a specific
column, as shown in \Cref{tab:example}.

\section{Figures}
Besides creating a separate floating environment for every Figure
(\Cref{fig:example}), multiple Figures can also be nicely grouped in sub
Figures, such as \Cref{fig:subfig1,fig:subfig2}.

\begin{figure}[t]
  \begin{center}
    \includegraphics[width=0.5\linewidth]{codesign}
  \end{center}
  \caption{This is an example figure.}
  \label{fig:example}
\end{figure}

\begin{figure}[t]
  \begin{center}
    \hfill
    \subfloat[First sub figure]{%
      \label{fig:subfig1}%
      \includegraphics[width=0.3\linewidth]{codesign}%
    }
    \hfill
    \subfloat[Second sub figure]{%
      \label{fig:subfig2}%
      \includegraphics[width=0.3\linewidth]{codesign}%
    }
    \hfill\null
  \end{center}
  \caption{This is an example for sub figures.}
  \label{fig:subfig}
\end{figure}

\pagebreak

\section{Listings}
A Listing can be put inline, like this

\lstinputlisting[%
  frame=none,%
  language=C++,%
  aboveskip=\medskipamount,%
  belowskip=\medskipamount]
  {code/helloworldshort.cpp}
which unfortunately cannot be referenced and could be split up into two on page
breaks. Alternatively, Listings could also be defined within a floating
environment, as provided in \Cref{lst:hello}.

\lstinputlisting[%
  float=t,%
  language=C++,%
  caption={Hello World example.},%
  label={lst:hello}]
  {code/helloworldfull.cpp}

It is also possible to include short code snippets like
\lstinline[language=C++]{std::vector<double>} in the text.

\section{Algorithms}
The equal sign (\code{=}) in pseudo code algorithms is used for testing
equality and not for assignments. For assignments, use \code{\textbackslash
gets} instead. Furthermore, variable names with more than one letter should be
set with \code{\textbackslash textrm}, as a name such as $var$ could be
interpret as $v\times a\times r$.

\begin{algorithm}
  \small
  \SetAlCapFnt{\small}
  \SetAlCapNameFnt{\small}

  \DontPrintSemicolon
  \SetKwProg{Func}{function}{}{end}
  \SetKwFunction{euclid}{Euclid}

  \caption{Euclid's algorithm}
  \label{alg:euclid}
  \Func(\tcp*[f]{The g.c.d. of a and b}){\euclid{$a,b$}}{
    $r \gets a\bmod b$\;
    \While(\tcp*[f]{We have the answer if r is 0}){$r\not=0$}{
      $a\gets b$\;
      $b\gets r$\;
      $r\gets a\bmod b$;\
    }
    \Return $b$ \tcp*{The gcd is b}
  }
\end{algorithm}

\section{Plots}
Plots should be defined as separate TikZ files. Data for plots should always be
provided in a separate data file. Rich examples for throughput, speedup, and
scatter plots are given in \Cref{fig:throughput,fig:speedup,fig:scatter}.

\begin{figure}[t]
  \begin{center}
    \tikzpicturedependsonfile{figures/throughput.tikz.tex}

\definecolor{codesignred}{cmyk}{0.000,1.000,1.000,0.200}

\begin{tikzpicture}[trim axis left,trim axis right]
  \pgfplotsset{
    width=.9\linewidth, 
    height=.5\linewidth,
    enlarge y limits=0.25,
  }

  \begin{semilogxaxis}[
      axis lines*=left,
      xbar,
      bar width=8pt,
      xmin=1,
      ytick=data,
      yticklabels = {
        \strut GB,
        \strut LP,
        \strut SB,
        \strut BL
      },
      major y tick style = {opacity=1},
      minor y tick style = {opacity=0},
      %y tick label style = {anchor=east, rotate=45},
      reverse legend,
      legend entries={
        GPU1,
        GPU2,
        GPU3
      },
      legend columns=3,
      legend style={at={(0,1)}, anchor=south west, draw=none, nodes={inner sep=3pt}},
      xlabel={Throughput [MPixel/s]}]

    \addplot+[xbar, color=codesignred!90] % GPU1
      table [x=GPU1, y=FILTER] {data/gpu.dat};
    \addplot+[xbar, color=codesignred!55] % GPU2
      table [x=GPU2, y=FILTER] {data/gpu.dat};
    \addplot+[xbar, color=codesignred!30] % GPU3
      table [x=GPU3, y=FILTER] {data/gpu.dat};

  \end{semilogxaxis}
\end{tikzpicture}

  \end{center}
  \caption{Example for throughput plot.}
  \label{fig:throughput}
\end{figure}

\begin{figure}[t]
  \begin{center}
    \tikzpicturedependsonfile{figures/speedup.tikz.tex}

\begin{tikzpicture}[trim axis left, trim axis right]
  \begin{axis}[%
      width=.9\linewidth,%
      height=.5\linewidth,%
      ybar=1pt,%
      bar width=7pt,%
      enlarge x limits=0.09,% horizontal spacing
      symbolic x coords={10,20,30,40,50,60,70},%
      xticklabels={{Simple},{Mandel-\\brot},{Stencil},{Laplace},{Gaussian},{Harris\\Corner},{Optical\\Flow}},%
      xticklabel style={align=center},%
      xtick=data,%
      xtick pos=left,
      ytick pos=left,
      y tick style={transparent},% hide x tick marks
      ymin=0,%
      ymax=7,%
      ylabel={Speedup},%
      x tick label style={font=\small},
      y tick label style={font=\small},
      ymajorgrids,
      ytick={0,2,4,6,8,10,12,14,16},
      legend style={at={(0,1.1)},draw=none,fill=none,anchor=north west,legend columns=6,font=\small},
      legend cell align=left]%

    \addplot[draw=RoyalBlue!80!black, fill=RoyalBlue!80!black]       table [y=ispc, x=app]     {data/vect.dat};
    \addplot[draw=LimeGreen!80!black, fill=LimeGreen!80!black]       table [y=icc, x=app]      {data/vect.dat};
    \addplot[draw=Goldenrod!80!black, fill=Goldenrod!80!black]       table [y=clang, x=app]    {data/vect.dat};
    \addplot[draw=YellowOrange!80!black, fill=YellowOrange!80!black] table [y=gcc, x=app]      {data/vect.dat};
    \addplot[draw=Red!80!black, fill=Red!80!black]                   table [y=baseline, x=app] {data/vect.dat};

    \legend{ISPC,ICC,Clang,GCC,Baseline},%
  \end{axis}
\end{tikzpicture}

  \end{center}
  \caption{Example for speedup plot.}
  \label{fig:speedup}
\end{figure}

\begin{figure}[t]
  \begin{center}
    \tikzpicturedependsonfile{figures/scatter.tikz.tex}

\begin{tikzpicture}[trim axis left,trim axis right]
  \tikzset{every pin/.style={pin distance=1em,font=\footnotesize}}
  \begin{axis}[
      width=.9\linewidth, 
      height=.5\linewidth,
      xlabel={Throughput [frame/s]},
      ylabel={Logic Utilization [\%]},
      legend style={font=\small},
      legend cell align=left,
      scatter/classes={%
        n={mark=*,gray!50},%
        y={mark=*,blue}}]

    \addplot[scatter, only marks, scatter src=explicit symbolic]
         table[x=FPS, y=Util, meta=Single] {data/pareto.dat}
         %node[pos=0/7, pin=right:NDRange]{}
         node[pos=1/7, pin=below right:CU2]{}
         node[pos=2/7, pin=right:CU4]{}
         node[pos=3/7, pin=right:SIMD2]{}
         node[pos=4/7, pin=below right:SIMD4]{}
         node[pos=5/7, pin=right:CU2/SIMD2]{}
         %node[pos=6/7, pin=right:Single]{}
         node[pos=7/7, pin=above left:Line Buffer]{};

    \legend{NDRange,Single}
  \end{axis}
\end{tikzpicture}

  \end{center}
  \caption{Example for scatter plot.}
  \label{fig:scatter}
\end{figure}

\section{Todos}
Todos\todo{This is a todo} can simply be set everywhere in text. They will
automatically be hidden in \code{final} and \code{print} mode.

